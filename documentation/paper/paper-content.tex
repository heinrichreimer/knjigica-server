\section{Einleitung}
Die Textvorlage einer Oper, das Libretto,
ist eine wichtige Quelle, sowohl für Besucher einer Oper,
sowie für die Literaturwissenschaft.
Für die schnelle Recherche von Libretti steht bisher
keine Suchmaschine zur Verfügung.
Wir entwerfen daher eine Suchmaschine, mit der sich Libretti schnell
anhand ihres Textes oder ihrer Eigenschaften finden lassen.

Durch diese neue Suchmaschine, sind zahlreiche Verbesserungen
bei der Arbeit mit Libretto zu erwarten: 
So ist es beispielsweise hilfreich, 
wenn der Text aus anderen Sprachen übersetzt wird, 
um ein besseres Verständnis für die Inhalte der Handlung zu bekommen.
Auch bekommen Besucher die Möglichkeit, 
mit Hilfe des Textes der Handlung besser folgen zu können
und nicht nur dem Geschehen auf der Bühne zuzuschauen.
Dies ist auch der Fall, 
wenn während der Aufführung ein bestimmter Teil nicht verstanden wurde,
weil die Präsentation oder Aussprache undeutlich war,
oder durch starke Betonung unverständlich wurde. 

Mit Hilfe der Suche im Libretto können des Weiteren
auch zusätzliche Informationen zum Autor
oder dem Jahr der Erstaufführung recherchiert werden,
was in der Vor- und Nachbereitung eines Besuches
einer Opernaufführung von Bedeutung ist. 

Die Möglichkeiten der Benutzung dieser Suchmaschine sind also vielfältig
und können sogar der Forschung in der Librettologie zugutekommen.

\section{Datenquellen}
\label{sec:corpora}
Im ersten Schritt der Umsetzung ist eine umfassende Recherche
nach Libretto und deren Texten im Internet erfolgt.
\begin{table}
    \centering
    \begin{tabular}{l r}
        \toprule
        Corpus & Anzahl Libretto \\
        \midrule
        Kareol & 385 \\
        Opera Lib & 434 \\
        Opera Folio & 512 \\
        Opera Glass & 170 \\
        Aria & 177 \\
        Collection Ulric Voyer & 220 \\
        \bottomrule
    \end{tabular}
    \caption{Libretto-Corpora}
    \label{tab:corpora}
\end{table}
Als Ergebnis der Suche wurden die in Tabelle \ref{tab:corpora} aufgeführten
Datenquellen ausgemacht.

Zeitgeschichtlich wurde der überwiegene Teil der
in den Corpora enthaltenen Libretti
im Zeitraum vom 17. bis 20. Jahrhundert uraufgeführt.
Die meisten der Corpora enthalten Libretti verschiedener Sprachen,
überwiegend jedoch italienische, franzöische,
englische, deutsche oder russische.
Auch werden bei einigen Corpora Übersetzungen zur Verfügung gestellt,
so werden beispielsweise die Libretti von \emph{Opera Glass}
durch 389 Übersetzungen ergänzt.
Bei \emph{Kareol} sogar die Übersetzung
dem Originaltext gegenübergestellt.

\section{Struktur}

\subsection{Literarische Struktur von Libretto}
Im geschichtlichen Verlauf der Zeit haben sich Libretti
stetig weiterentwickelt.
Daher ist es schwierig, für ein Libretto eine feste Struktur festzulegen.
Allgemein sin jedoch einige Bestandteile in jedem Libretto vorhanden.
So wird anfangs der Titel und die Autoren genannt.
Hier werden auch oftmals weitere Informationen, wie Daten zur Premiere,
sowie weitere Notizen oder Widmungen des Autors genannt.
Nachfolgend werden die im Libretto vorkommenden Rollen mit Namen,
teils mit Beschreibung, sowie mit der vorgesehenen Stimmlage, aufgeführt.
Anschließend folgt die Handlung des Libretto.
So werden die Überschriften, sowie Unterüberschriften genannt.
Textpassagen werden, meist mit dem Namen der vortragenden Rolle versehen,
einfach aufgelistet. Zwischen Textpassagen stehen gelegentlich
Regieanweisungen zur Betonung, Informationen zum Bühnenbild,
sowie weitere Instruktionen.
Mit der Handlung endet das Libretto.

\subsection{Hilfsdatenstruktur}
Aus der allgemeinen Struktur von Librettos
leiten wir für die Entwicklung der Suchmaschine
ein standardisiertes und maschinell lesbares Format ab.
Dieses dient dem effizienten Speichern der gerawlten Rohdaten
und der Vorbereitung sowie Standardisierung
für die spätere erfolgende Indizierung durch die Suchmaschine.

Neben einer Datenstruktur, die ein vollständiges Libretto abbildet,
entwerfen wir Klassen, die einzelne Teile der Struktur speichern.
In einer Datenstruktur werden alle Informationen zum Autor gespeichert,
in einer zweiten Informationen über Widmungen und Notizen des Autors,
in einer dritten Informationen zu einer Rolle bzw. einem Darsteller,
und in einer letzten werden alle
die Handlung betreffenden Informationen abgespeichert.
Dabei wird diese letzte in sich in drei Datenstrukturen unterteilt,
für Textpassagen, Instruktionen, sowie für Überschriften.

\section{Technische Implementierung}
Die Implementierung der Libretto-Suchmaschine unterteilt sich
in mehrere Bausteine.
Während der Crawler vor allem auf Linux-Bash-Programmen aufbaut,
wird in den anderen Bausteinen, so auch im Parser, Indexer und im Benutzerprogramm
auf objektorientierte Programmierung mittels Java/Kotlin gesetzt.
Als Framework wird das praxiserprobte \emph{Elasticsearch}-Framework verwendet.
Im Folgenden werden die einzelnen Komponenten der Suchmaschine
näher erklärt.

\subsection{Crawler}
Der von uns entwickelte Crawler hat die Aufgabe,
die Corpora in \ref{sec:corpora} aus dem Internet herunterzuladen,
und für die weitere Verarbeitung zu speichern.

Zuerst werden dazu die URLs aller Libretti aus dem Register
eines jeden Corpus extrahiert und als Liste zwischengespeichert.
Um Probleme mit relativen URLs auszuschließen,
werden diese auf die absolute Adresse erweitert.

Danach lädt ein weiteres Skript systematisch die HTML-Dokumente
aller erfassten URLs herunter
und speichert diese zunächst unverändert ab.
Hierbei werden auch verlinkte, verschachtelte Seiten zunächst mit übernommen.

Beide Vorgänge wurden mit Hilfe von Bash-Skripten umgesetzt,
die die Linux-Funktionen \code{wget}, \code{curl},
sowie die \code{html-xml-utils} verwenden.

\subsection{Parser}
Die Inhalte der Internetseite liegen nach dem Crawlen
als HTML-Quelltext vor.
Damit in der Folge nach enthaltenen Informationen gesucht werden kann,
werden die Inhalte mittels Java bzw. Kotlin in ein Objekt geschrieben.
Hierbei wird die Bibliothek \emph{JSoup} verwendet,
welche die Extraktion und Bearbeitung von HTML-Inhalten ermöglicht.
Entsprechend werden die gecrawlten Dokumente je nach Quelle
so verallgemeinert, dass die Hilfsdatenstruktur anhand der Inhalte
spezifischer HTML-Elemente erzeugt wird.

Zur besseren Speicherung der so erhaltenen Objekte
für die nächsten Bearbeitungsschritte,
werden zusätzlich ein Parser und ein Formatter entwickelt,
mit dem die in der Hilfsdatenstruktur enthaltenen Informationen
in das JSON-Format umgewandelt,
sowie aus diesem wieder geparst werden können.

\subsection{Suchmaschine}
Als Suchmaschine verwenden wir die Open-Source-Software
\emph{Elasticsearch}\footnote{\url{https://www.elastic.co/products/elasticsearch}}.
Diese beruht auf dem \emph{Lucene}-Framework\footnote{\url{http://lucene.apache.org/}} von Apache
und realisiert die Kommunikation mit Klienten
über ein RESTful-Webinterface.

\subsection{Indizieren}

Da das \emph{Elasticsearch}-Framework
keine verschachtelten Arrays aus Objekten handhaben kann,
werden stattdessen angepasste Datenstrukturen verwendet,
die durch Verflachen aus der umfassenden Datenstruktur entsteht.
So wird beispielsweise in einem Handlungs-Objekt
noch der Titel der Oper mit gespeichert,
und die Information zur aktuellen Überschrift.

Auch für diese Strukturen werden Formatter in das JSON-Format erstellt.

Die aus den zwischengespeicherten JSON-Daten geparsten Dokumente
werden nun aufgespalten, sodass aus einem Libretto eine Vielzahl von
Teilobjekten entsteht.
Diese werden dann jeweils nach Typ in einzelne Indices
von \emph{Elasticsearch} über das REST-Interface des Clusters indiziert.

\subsection{Anfrageverarbeitung}

Im Anschluss liegen die Dokumente aufgelistet in \emph{Elasticsearch} vor
und können über den Index von der Suchmaschine gefunden werden.

Eine Suchanfrage wird, wie die Inhalte zuvor,
im JSON-Format über das REST-Interface an das \emph{Elasticsearch}-Cluster gesendet.
Die Anfrage kann verschiedene Parameter enthalten,
welche die Resultate mit Filtern versehen
und Spezifikationen wie eine bestimmte Ordnung der Inhalte
und Eingrenzung auf bestimmte Stichworte erlauben.

Für die Libretto-Suchmaschine werden mehrere Indices,
nämlich die der verschiedenen Dokumenttypen,
gleichzeitig angefragt.
Dabei werden die Ergebnisse je nach Typ unterschiedlich gewichtet,
sodass bei Konflikten beispielsweise Resultate,
den Titel der Oper betreffend,
stärker gewichtet werden können,
als solche, die in einer Textpassage vorkommen.

Durch die Kombination aller Indices kann über das selbe Suchfeld nach
Titel, Autor, Rollennamen, Regieanweisungen, Akt, Szene,
Text, Widmung und Premiere gesucht werden.

\section{Evaluation}
Die Evaluation wurde anhand des Vorbildes der
„Text Retrieval Conference“ (TREC) durchgeführt.
In diesem Rahmen wird die Suchmaschine anhand von 25 Topics
auf ihre Funktionalität untersucht.

Ein Topic besteht etwa aus einer zentralen Suchanfrage,
aus der weitere Unteranfragen resultieren können.
Um wichtige Aspekte der Suchmaschine abzudecken,
werden die Topics nach den Typen \textquote{Autor},
\textquote{Struktur} und \textquote{Handlung} eingeordnet.

Die Relevanz der Suchanfragen werden anhand einer Skala von 0 bis 1 erfasst:
Hierbei steht ein Wert von 0 für ein Suchergebnis,
welches überhaupt nicht relevant oder nicht hilfreich ist,
ein Wert von 1 für ein Ergebnis, das das Informationsbedürfnis des Benutzers
aus der Suchanfrage in vollem Umfang erfüllt.

\subsection{Durchführung}

Einem unabhängigen Dritten wurden die Topics präsentiert,
und dieser konnte selbstständig über die Suchbox
eine Anfrage -- oder die vorgeschlagene Beispielanfrage -- eingeben.

Nach Durchführung einer jeden Suchanfrage sollte die Testperson
die Relevanz anhand der oben beschriebenen Skala einschätzen
und zusammen mit einem optionalen Kommentar erfassen,
welcher Aufschluss über die Einordnung der vorgegebenen Skala gibt.

Die Resultate wurden zusätzlich
mit der aktuellen Version der Suchmaschine annotiert,
sodass ein Vergleich verschiedener Versionen der Suchmaschine möglich ist.

\subsection{Ergebnis}

Im Mittel ergibt sich über alle Topics eine Relevanz von 0,344,
was eine verhältnismäßig geringe Erfüllung der Suchanfragen bedeutet.
\begin{table}
    \centering
    \begin{tabular}{l r r}
        \toprule
        Typ & Anzahl Topics & Durchschnittliche Relevanz \\
        \midrule
        Autor & 9 & 0,510 \\
        Struktur & 6 & 0,000 \\
        Handlung & 10 & 0,400 \\
        alle & 25 & 0,344 \\
        \bottomrule
    \end{tabular}
    \caption{Relevanz nach Topic-Typ}
    \label{tab:rel-topic-type}
\end{table}
Jedoch fällt die Relevanz
bei den unterschiedlichen Topic-Typen verschieden aus,
wie in Tabelle \ref{tab:rel-topic-type} ersichtlich ist.
Die Typen \textquote{Autor} und \textquote{Handlung}
liefern also beide bessere Suchergebnisse,
was durch den derzeitigen Stand der Umsetzung zu begründen ist.

Die Suchmaschine ist in der Lage, nach spezifischen Informationen zu suchen,
berücksichtigt jedoch noch keine Operatoren oder Suchanfragen,
welche eine Umwandlung der Inhalte
anhand einer bestimmten Fragestellung enthält.
Dies führt auch dazu, dass die Intention einer Suchanfrage
nicht vollständig korrekt erkannt wird
und die eigentlich richtig eingeordneten Suchergebnisse
mit falschen Parametern ausgegeben werden.

\section{Zusammenfassung \& Ausblick}
Im derzeitigen Umsetzungsstand der Suchmaschine
können innerhalb einer Kommandozeilenanwendung Anfragen gestellt werden,
welche die bereits vollständig indizierten Dokumente
nach passenden Inhalten durchsucht.
Dabei werden als Resultate nur übereinstimmende Inhalte ausgegeben,
welche in den Kategorien „Titel“ und „Sprache“ (3), „Autor“ (1,5),
„Rollen“ (1,2), „Bemerkungen“ und „Widmungen“ (0,7), „Plot“ (0,15)
mit absteigenden Gewichtungen festgelegt sind.

Im Suchprozess wird auf eine Datenbank zugegriffen,
welche Dokumente mit einer von uns festgelegten Datenstruktur enthält.
Diese wurde im jetzigen Stand lediglich aus einer Datenquelle heraus befüllt.

\subsection{Corpus}
Obwohl wir bisher nur für den \emph{Opera Lib}-Corpus
einen Parser geschrieben haben,
lässt sich das entwickelte Framework problemlos auf weitere Corpora erweitern.
So planen wir beispielsweise,
die im Gliederungspunkt „Datenquellen“ angegebenen Seiten zu crawlen
und in die Hilfsdatenstruktur zu transferieren.
Dies setzt auch die Zusammenführung
mit den bereits vorhandenen Dokumenten voraus,
welche als Folge eventuell doppelt
und nicht mehr redundanzfrei vorliegen würden.
Als Folge dieses Prozesses dürften Libretto teilweise
in mehreren Sprachen vorliegen
und könnten auch in vollständigerer Form hinterlegt sein.

Eine weitere Erweiterungsmöglichkeit besteht darin,
Kunstlieder mit in die Suchmaschine einzubeziehen.
Der \emph{Kareol}-Corpus stellt dazu 687 Lieder zur Verfügung.

\subsection{Query Understanding}
Ein weiteres Ziel in folgenden Versionen soll es sein,
die Suchanfragen zu analysieren
und somit auch auf eventuelle Spezifizierungen
eine angepasste Ausgabe geben zu können.
Ein wichtiger Punkt wäre hierbei zunächst
das sogenannte „Natural Language Processing“ \cite{liddy2001natural},
welches mit Hilfe der gegebenen Quellen Sprachen erkennt,
deren Inhalte segmentiert und im Anschluss Funktionen
und Bedeutungen der Wörter herausfiltert.
Mit diesem Vorgehen würde ein hauptsächlicher Grund
für die schlechten Ergebnisse der Evaluation entfallen.
Zusätzlich zur sprachlichen Verarbeitung der Suchanfragen
wäre eine mathematische Anwendung denkbarbar,
welche Anfragen bezüglich bestimmter Mengen weiterverarbeitet.
Gemeint ist hiermit zum Beispiel ein Rechner,
welcher es ermöglicht die Anzahl bestimmter Worte oder Absätze
in einem Dokument zu berechnen. Welche Form diese Anwendung haben
könnte bleibt zum jetzigen Zeitpunkt noch offen und kann beliebig gewählt werden.
Die Umsetzung beider Anwendungen würde zumindest bezogen auf die in dieser
Dokumentation durchgeführten Evaluation gestellten Querys ein passendes Resultat liefern.
