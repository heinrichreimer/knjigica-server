\documentclass[
    sigconf, % Conference paper style
    % timestamp, manuscript, %authordraft, % For writing
    natbib=false, % Disable natbib package, we'll use biblatex instead.
    nonacm=true % Disable ACM-related content.
]{acmart}


% Kodierung:
\usepackage[utf8]{inputenc} % UTF8-Dateikodierung
\usepackage[T1]{fontenc} % Ausgabekodierung


% Letter paper
\geometry{a4paper} % Use A4 paper format instead of US Letter.


% Grundlegende Support-Pakete:
\usepackage[ngerman]{babel} % Sprachunterstützung für korrekte Silbentrennung und Übersetzung der Standard-LaTeX-Texte
\usepackage{hyperref} % Querverweise für externe Webseiten und interne Abschnitte/Abbildungen/Quellenangaben etc.
\usepackage{graphicx} % externe Bilder/Grafiken
\usepackage{booktabs} % More formal table style.
\usepackage[inline]{enumitem} % Inline enumerations (looks better in acmart)

% Bibliography & Cites
\usepackage[autostyle=true]{csquotes} % Anführungszeichen und direkte Zitate
\usepackage[
    abbreviate=true,
    dateabbrev=true,
    isbn=true,
    doi=true,
    urldate=comp,
    url=true,
    maxbibnames=9,
    backref=false,
    backend=biber,
    style=alphabetic,
]{biblatex} % Load biblatex and set ACM styles.
\renewcommand{\bibfont}{\Small}
\addbibresource{bibliography.bib} % Load bibliography file.


% Glossaries
\usepackage{glossaries}
\makeglossaries


% Schrift:
% \usepackage{eulervm} % Schriftart "Euler Maths" für Mathematik
\usepackage{eurosym} % offizielles Euro-Zeichen


% Schriftformatierung:
\newcommand{\strong}[1]{\textbf{#1}} \newcommand{\fett}[1]{\textbf{#1}} % fetter Text
\newcommand{\italic}[1]{\textit{#1}} \newcommand{\kursiv}[1]{\textit{#1}} % kursiver Text
\newcommand{\code}[1]{\texttt{#1}} % monospace Text
\newcommand{\unterstrichen}[1]{\underline{#1}} % unterstrichener Text
\newcommand{\Underline}[1]{\underline{\underline{#1}}} \newcommand{\Unterstrichen}[1]{\Underline{#1}} % doppelt unterstrichener Text
\newcommand{\plural}[1]{\textsuperscript{\underline{$#1$}}} \newcommand{\pl}[1]{\plural{#1}} % Abgekürzter Plural (höhergestellt und unterstrichen) für "Lineare Algebra"
\newcommand{\numberlinefont}{\tiny} % Schriftart von Zeilennummern bei Algorithmen und Listings
\newcommand{\numberlineskip}{0.75em} % Abstand von Zeilennummern bei Algorithmen und Listings
\newcommand{\queryfont}{\footnotesize\ttfamily} % Query text style
\newcommand{\query}[1]{{\queryfont#1}} % Query text style


% Code-Formatierung:
\usepackage{listings} % Paket für Code-Listings
\usepackage{listingsutf8} % UTF8-Support in importierten Listings
\lstset{
    language=C, % Standardsprache
    basicstyle=\ttfamily, % Schriftart
    breakatwhitespace=false, % Zeilenumbruch nur bei Leerzeichen
    breaklines=true, % Automatischer Zeilenumbruch
    prebreak={\mbox{$\hookleftarrow$}}, % Zeichen zur Kennzeichnung autom. Zeilenümbrüche
    numbers=left,numberstyle=\numberlinefont,numbersep=\numberlineskip,stepnumber=1, % Zeilennummern
    morecomment=[s][]{/**}{*/}, % Kommentare
    keepspaces=true, % Leerzeichen übernehmen
    keywordstyle=\bfseries, % Schlüsselwörter
    showtabs=false, showspaces=false, % Tabs und Leerzeichen nicht kennzeichnen
    showstringspaces=false, % Leerzeichen in Zeichenketten nicht kennzeichnen
    tabsize=4, % Tabulatorgröße
}
\lstdefinestyle{haskell}{ % Quellcode-Stil für Haskell ("Konzepte der Programmierung")
    language=haskell,
    escapeinside={*'}{'*},
    showstringspaces=false,
    morecomment=[l]\%,
    captionpos=b,
    emphstyle={\bfseries},
    tabsize=2,
}
\lstdefinelanguage{json}{
    morestring=[b]",
    morestring=[d]'
    literate=
     *{0}{{{\color{numb}0}}}{1}
      {1}{{{\color{numb}1}}}{1}
      {2}{{{\color{numb}2}}}{1}
      {3}{{{\color{numb}3}}}{1}
      {4}{{{\color{numb}4}}}{1}
      {5}{{{\color{numb}5}}}{1}
      {6}{{{\color{numb}6}}}{1}
      {7}{{{\color{numb}7}}}{1}
      {8}{{{\color{numb}8}}}{1}
      {9}{{{\color{numb}9}}}{1}
      {:}{{{\color{punct}{:}}}}{1}
      {,}{{{\color{punct}{,}}}}{1}
      {\{}{{{\color{delim}{\{}}}}{1}
      {\}}{{{\color{delim}{\}}}}}{1}
      {[}{{{\color{delim}{[}}}}{1}
      {]}{{{\color{delim}{]}}}}{1},
}


% Algorithms and pseudocode
\usepackage[vlined, linesnumbered]{algorithm2e} % Algorithms and pseudocode package
\DontPrintSemicolon % Hide semicolons
\SetKwProg{Function}{function}{\ is}{end function} % Function definition
\SetKwComment{Comment}{\quad$\triangleright$\ }{}\SetCommentSty{itshape} % Comment style
\SetKw{Continue}{continue}
\SetKwBlock{Repeat}{repeat}{}
\SetNlSty{numberlinefont}{}{} % Zeilennummer-Schrift
\SetNlSkip{\numberlineskip} % Zeilennummer-Abstand
\SetAlgoNlRelativeSize{0}
\SetAlFnt{\footnotesize}


% Grafiken und Bilder
\usepackage{rotating} % Rotieren
\usepackage{tikz} % Tikz Graphen
\tikzset{>=latex} % Tikz Grundkonfiguration
\usepackage{pgfplots} % Funktions-Plots in 2D/3D
\pgfplotsset{compat=1.16}
    

% Mathematik
\usepackage{amsthm,amsmath,amssymb,amstext,MnSymbol} % Grundpakete für Matheumgebungen, Sonderzeichen, Brüche etc. (MnSymbol kann auch weggelassen werden)
\usepackage{array} % Tabellen in Mathematik-Umgebungen
\usepackage{cancel} % durchgestrichener Text (z.B. \not\in)
\newcommand{\union}{\cup} \newcommand{\vereinigung}{\cup} % Vereinigung
\newcommand{\disjunction}{\uplus} % disjunkte Vereinigung
\newcommand{\intersection}{\cap} \newcommand{\intersect}{\cap} \newcommand{\schnitt}{\cap} % Schnittmenge
\newcommand{\lund}{\land} % Logisches Und
\newcommand{\loder}{\lor} % Logisches Oder
\newcommand{\mland}{\(\land\)} \newcommand{\mlund}{\mland} % Logisches Und im Fließtext
\newcommand{\mlor}{\(\lor\)} \newcommand{\mloder}{\mlor} % Logisches Oder im Fließtext
\renewcommand{\C}{\mathbb{C}} \newcommand{\complexnumbers}{\C} \newcommand{\complexezahlen}{\C} % Shortcut für Mengensymbol der komplexen Zahlen
\newcommand{\R}{\mathbb{R}} \newcommand{\realnumbers}{\R} \newcommand{\reelezahlen}{\R} % Shortcut für Mengensymbol der reelen Zahlen
\newcommand{\Q}{\mathbb{Q}} \newcommand{\rationalnumbers}{\Q} \newcommand{\rationalezahlen}{\Q} % Shortcut für Mengensymbol der rationalen Zahlen
\newcommand{\Z}{\mathbb{Z}} \newcommand{\wholenumbers}{\Z} \newcommand{\ganzezahlen}{\Z} % Shortcut für Mengensymbol der ganzen Zahlen
\newcommand{\N}{\mathbb{N}} \newcommand{\naturalnumbers}{\N} \newcommand{\natuerlichezahlen}{\N} % Shortcut für Mengensymbol der natürlichen Zahlen
\newcommand{\B}{\mathbb{B}} \newcommand{\binarynumbers}{\B} \newcommand{\binaerezahlen}{\B} % Shortcut für Mengensymbol der binären Zahlen
\renewcommand{\O}{\mathcal{O}} % asymptotische Laufzeit in O-Notation ("Datenstrukturen und effiziente Algorithmen")
\newcommand{\base}[1]{\mathcal{#1}} % Formatierung für Basis ("Lineare Algebra")
\newcommand{\rechtspfeil}{\rightarrow} % Shortcut für Pfeil nach rechts
\newcommand{\Rechtspfeil}{\Rightarrow} % Shortcut für doppelten Pfeil nach rechts
\newcommand{\linkspfeil}{\leftarrow} % Shortcut für Pfeil nach links
\newcommand{\Linkspfeil}{\Leftarrow} % Shortcut für doppelten Pfeil nach links
\newcommand{\linksrechtspfeil}{\leftrightarrow} % Shortcut für Pfeil nach links und rechts
\newcommand{\Linksrechtspfeil}{\Leftrightarrow} % Shortcut für doppelten Pfeil nach links und rechts
\DeclareMathOperator{\im}{im} % Bild
\DeclareMathOperator{\id}{id} % Identität
\DeclareMathOperator{\sel}{sel} % Selektion
\DeclareMathOperator{\dom}{dom} % Definitionsbereich
\DeclareMathOperator{\ran}{ran} % Wertebereich
\DeclareMathOperator{\Hom}{Hom} % Homomorphismus
\DeclareMathOperator{\End}{End} % Endomorphismus
\DeclareMathOperator{\indeg}{indeg} % Eingangsgrad
\DeclareMathOperator{\outdeg}{outdeg} % Ausgangsgrad
\let\tmp\mod \let\mod\bmod \let\bmod\tmp % Tausche \mod und \bmod Befehle
\let\varemptyset\emptyset \let\emptyset\varnothing % Verbesserte Darstellung der leeren Menge \emptyset (\varemptyset für alte Darstellung)
\let\tmp\epsilon \let\epsilon\varepsilon \let\varepsilon\tmp % Verbesserte Darstellung von Epsilon (Tausche \epsilon und \varepsilon)
\let\tmp\phi \let\phi\varphi \let\varphi\tmp % Verbesserte Darstellung von Phi (Tausche \phi und \varphi)
\newcommand{\eqtransform}{\ensuremath{\qquad\big|\,\,}} % Äquivalenzumformung
\newcommand{\infinity}{\infty} % Unendlich-Zeichen
\newcommand{\corresponds}{\triangleq} % Entspricht-Zeichen
\newcommand{\ditto}{\textquotedbl} \newcommand{\dito}{\ditto} % "Siehe oben"-Zeichen
\renewcommand{\qed}{\nopagebreak\hfill\ensuremath{\square}} % Ende eines Beweises: "quod erat demonstrandum", lat. "was zu beweisen war"
\newcolumntype{L}{>{\(}l<{\)}} % linksbündige Tabellenspalte im Mathe-Modus
\newcolumntype{R}{>{\(}r<{\)}} % rechtbündige Tabellenspalte im Mathe-Modus
\newcolumntype{C}{>{\(}c<{\)}} % zentrierte Tabellenspalte im Mathe-Modus

